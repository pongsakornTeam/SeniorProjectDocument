\chapter{บทนำ}
\label{intro}
%%%%%%%%%%%%%%%%%%%%%%%%%%%%%%%%%%%%%%%%%%%%%%%%%%%%%%%%%%%%%%%%%%%%%%%%%%%%%% 
%
\section{ที่มาและความสำคัญ}
ในปัจจุบันมีการบริการมากมายเกิดขึ้นในชีวิตประจำวันของมนุษย์ ซึ่งการบริการในการขายสินค้าอย่างหนึ่งต้องมีการพูดคุยหรือติดต่อกับลูกค้า ซึ่งหากจะกล่าวถึงการบริการแบบร้านค้าที่มีพนักงานขายติดต่อสื่อสารกับลูกค้าที่มารับบริการโดยตรงแล้ว ความพึงพอใจการบริการที่ได้รับ ก็สามารถแสดงออกในรูปแบบของ สีหน้าอารมณ์ คำพูด และกริยาท่าทางหลังจากได้รับบริการ

ความพึงพอใจต่อการบริการ จึงเป็นสาระสำคัญในการขายสินค้าหรือบริการ ซึ่งสามารถวัดระดับคุณภาพการบริการได้จากลูกค้าที่กำลังได้รับการบริการอยู่ในขณะนั้นเอง  ซึ่งเป็นเหตุผลที่โครงงานนี้เสนอการสร้างซอฟต์แวร์ที่สามารถตรวจจับอารมณ์ของลูกค้า ซึ่งถือว่าเป็นสิ่งที่ระบุความพึงพอใจของลูกค้าได้ในระดับหนึ่ง โดยวิธีการทำงาน คือ นำข้อมูลภาพเคลื่อนไหวที่ได้รับจากกล้องวงจรปิดภายในร้านค้า มาใช้เป็นแหล่งข้อมูลเพื่อให้ซอฟต์แวร์วิเคราะห์ข้อมูลนั้น แล้วบันทึกผลลัพธ์ออกมาซึ่งเป็นความพึงพอใจของลูกค้าทุกคนที่ปรากฎในภาพเคลื่อนไหวจากกล้องวงจรปิด ณ เวลาขณะนั้นได้  และในการที่จะยกระดับหรือปรับปรุงการบริการ จำเป็นจะต้องมีการบันทึกข้อมูลการบริการของพนักงาน เพื่อให้ทราบว่าลูกค้ากำลังได้รับการบริการจากพนักงานคนใดและมีระดับความพึงพอใจต่อการบริการในระดับใดด้วย

เมื่อมีการใช้งานซอฟต์แวร์ที่ทำการวิเคราะห์ความพึงพอใจของลูกค้าแล้ว จึงต้องมีการใช้ภาพเคลื่อนไหวจากกล้องวงจรปิดเพื่อเป็นแหล่งข้อมูลขาเข้าให้ซอฟต์แวร์วิเคราะห์ ซึ่งภาพเคลื่อนไหวที่ได้จะต้องมีรูปใบหน้าของลูกค้าอย่างชัดเจน โครงงานนี้จะพัฒนาต่อยอดซอฟต์แวร์ที่สามารถวิเคราะห์ได้ทีละหนึ่งภาพเคลื่อนไหวที่เป็นข้อมูลขาเข้าจากกล้องวงจรปิดเพียงหนึ่งมุมเท่านั้น ซึ่งโครงงานนี้จะอ้างอิงจากวิทยานิพนธ์และซอฟต์แวร์ของคุณ Nabil Tahmidul Karimv \cite{nabil} และSanjana Jain \cite{sanjana} โดยจะทำการพัฒนาและประยุกต์ให้ใช้สามมุมกล้องภายในร้านค้าในการวิเคราะห์ภาพเคลื่อนไหวทั้งสามมุม เพื่อตรวจจับอารมณ์ที่แสดงออกความพึงพอใจของลูกค้าที่มีต่อพนักงานขายที่กำลังบริการอยู่ ณ ขณะนั้น โดยใช้เวลาเป็นตัวเชื่อมโยงความสัมพันธ์ของภาพเคลื่อนไหวทั้งสามมุมจากสามกล้องวงจรปิดในเวลาเดียวกัน

เนื่องจากการทำงานของซอฟต์แวร์ที่ถูกพัฒนาขึ้นมานั้นสามารถจดจำลูกค้าและตรวจจับความพึงพอใจจากอารมณ์ โครงงานนี้จึงมีการนำมาพัฒนาระบบให้ตรวจจับความพึงพอใจของลูกค้าต่อพนักงานบริการ จากภาพเคลื่อนไหวหลายจากกล้องวงจรปิดหลายมุมขึ้นมา เพื่อทำให้สามารถใช้เป็นข้อมูลในปรับปรุงการบริการได้อย่างมีประสิทธิภาพและเป็นการนำความก้าวหน้าทางเทคโนโลยีมาใช้ประโยชน์ในทางการค้าอย่างมีประสิทธิภาพ โดยนำทฤษฎีการประมวลผลภาพดิจิตอล (digital image processing)  และการเรียนรู้ของเครื่องจักร (machine learning)  ในประเภทการเรียนรู้แบบกำกับดูแล (supervised leaning)  ซึ่งเป็นสาขาหนึ่งของปัญญาประดิษฐ์ (artificial intelligence)  ที่พัฒนามาจากการศึกษารู้จำรูปแบบ เพื่อสร้างขั้นตอนวิธีที่สามารถประมวลผลภาพเคลื่อนไหวที่เกิดจากการนำภาพดิจิตอลหลายๆภาพมาเรียงต่อกันให้สามารถแสดงผลลัพธ์ที่ต้องการได้ นั่นคือ  ระดับความพึงพอใจของลูกค้าต่อพนักงานบริการที่กำลังบริการอยู่
 
\section{วัตถุประสงค์}
\begin{enumerate}[label=1.2.\arabic*]
%\setlength{\itemindent}{20pt}
\item{เพื่อศึกษาทฤษฎีการวิเคราะห์และประมวลผลภาพเคลื่อนไหวที่มีภาพบุคคลคนเดียวกันที่ปรากฏพร้อมกันในกล้องวงจรปิด เพื่อออกแบบระบบให้มีความสามารถในการตรวจจับและระบุว่าเป็นบุคคลเดียวกัน}
\item{เพื่อพัฒนาขั้นตอนวิธีเพื่อให้ตรวจจับความพึงพอใจของลูกค้าที่มีต่อพนักงานที่กำลังบริการในขณะนั้น}
\end{enumerate}

\section{ขอบเขตการดำเนินงาน}
\begin{enumerate}[label=1.3.\arabic*]
%\setlength{\itemindent}{20pt}
\item{กลุ่มตัวอย่างที่ใช้ในการวิจัย ได้แก่ ภาพเคลื่อนไหวที่ได้จากกล้องวงจรปิดที่ร้าน
  Hom Krun สาขา AIT}
\item{ออกแบบและพัฒนาขั้นตอนวิธีให้สามารถตรวจจับความพึงพอใจของลูกค้าที่มีต่อพนักงานที่กำลังให้บริการในขณะนั้นได้}
\end{enumerate}

\section{ขั้นตอนการดำเนินงาน}
\begin{enumerate}[label=1.4.\arabic*]
%\setlength{\itemindent}{20pt}
\item{ศึกษาทฤษฎีเกี่ยวกับการรู้จำของเครื่อง (machine learning)}
\item{เก็บข้อมูลภาพเคลื่อนไหวจากกล้องวงจรปิดที่ร้าน
  Hom Krun สาขา AIT}
\item{วิเคราะห์ภาพเคลื่อนไหวและนำภาพใบหน้าพนักงานมาทำเป็นภาพนิ่งของแต่ละบุคคลเพื่อนำมาหาลักษณะเด่น (features)}
\item{ทำการนำภาพนิ่งที่มีรูปใบหน้าของพนักงานเพื่อใช้เป็นข้อมูลเข้าและลักษณะเด่น เพื่อมาทำการฝึกฝนพัฒนาระบบที่สามารถจำแนกประเภทด้วยการเรียนรู้แบบกำกับดูแล (supervised learning classifier)}
\item{สกัดลักษณะเด่นให้มีประสิทธิภาพสูงสุดโดยวิเคราะห์ทั้งความถูกต้องและแม่นยำ (accuracy)}
\item{สร้างระบบที่มีการระบุตัวตนของบุคคลคนเดียวกันจากหลายภาพเคลื่อนไหว}
\item{เก็บข้อมูลความพึงพอใจของลูกค้าที่แสดงออกต่อพนักงานคนหนึ่งๆ}
\end{enumerate}

\section{ผลที่คาดว่าจะได้รับ}
\begin{enumerate}[label=1.5.\arabic*]
%\setlength{\itemindent}{20pt}
\item{ได้ขั้นตอนวิธีและซอฟท์แวร์ในการจดจำพนักงานและวิเคราะห์ความพึงพอใจของลูกค้าระหว่างบริการได้}
\item{ช่วยในการปรับปรุงหรือยกระดับการบริการให้มีประสิทธิภาพและรวดเร็วมากยิ่งขึ้น}
\end{enumerate}
\newpage
\begin{landscape}
\section{ตารางการดำเนินงาน}
\begin{table}[h!]
 \centering
  \begin{tabular}{|l|c|c|c|c|c|c|c|c|c|c|c|c|c|c|c|c|c|c|c|c|}
  
    \hline
    \multirow{2}{*}{\textbf{แผนการดำเนินงาน}}
    & \multicolumn{4}{c|}{สิงหาคม} 
    & \multicolumn{4}{c|}{กันยายน} 
    & \multicolumn{4}{c|}{ตุลาคม}
    & \multicolumn{4}{c|}{พฤศจิกายน}
    & \multicolumn{4}{c|}{ธันวาคม}\\
    \cline{2-21}
    & 1 & 2 & 3 & 4
    & 1 & 2 & 3 & 4
    & 1 & 2 & 3 & 4
    & 1 & 2 & 3 & 4
    & 1 & 2 & 3 & 4\\
    \hline
    
    ศึกษาทฤษฎีเกี่ยวกับ &x&x&x&x&&&&&&&&&&&&&&&& \\
    การรู้จำของเครื่อง  &&&&&&&&&&&&&&&&&&&&\\
    \hline
    พัฒนาโปรแกรมให้สามารถ  &&&&&x&&&&&&&&&&&&&&&\\
    ตรวจจับใบหน้าของลูกค้า &&&&&&&&&&&&&&&&&&&&\\
     \hline
     ศึกษาเกี่ยวกับ  &&&&&&x&x&x&x&&&&&&&&&&&\\
     camera calibration &&&&&&&&&&&&&&&&&&&&\\
    \hline
    ศึกษาเกี่ยวกับ &&&&&x&x&x&x&x&x&x&x&x&&&&&&&\\
    caffe model &&&&&&&&&&&&&&&&&&&&\\
    \hline
    พัฒนาโปรแกรมให้สามารถจดจำ  &&&&&&&&&&&&&&x&x&x&x&&&\\
    ลักษณะของพนักงาน  &&&&&&&&&&&&&&&&&&&&\\
    \hline
    ทำรายงาน &&&&&x&x&x&x&x&x&x&x&x&x&x&x&x&x&x&x\\
    \hline
    
  \end{tabular}
  \caption{ตารางการดำเนินโครงงาน}
  \label{fig:table}
\hrulefill
\end{table}
%\end{landscape}
%\newpage
%%%%%%%%%%%%%%%%%%%%%%%%%%%%%%%%%%%%%%%%%%%%%%%%%%%%%%%%
%\begin{landscape}
\begin{table}[h!]
 \centering
  \begin{tabular}{|l|c|c|c|c|c|c|c|c|c|c|c|c|c|c|c|c|c|c|c|c|c|}
  
    \hline
    \multirow{2}{*}{\textbf{แผนการดำเนินงาน}}
    & \multicolumn{4}{c|}{มกราคม} 
    & \multicolumn{4}{c|}{กุมภาพันธ์} 
    & \multicolumn{4}{c|}{มีนาคม}
    & \multicolumn{4}{c|}{เมษายน}
    & \multicolumn{4}{c|}{พฤษภาคม}\\
    \cline{2-21}
    & 1 & 2 & 3 & 4
    & 1 & 2 & 3 & 4
    & 1 & 2 & 3 & 4
    & 1 & 2 & 3 & 4
    & 1 & 2 & 3 & 4\\
	 \hline
    ศึกษาวิธีการวิเคราะห์อารมณ์ของลูกค้า  &x&x&x&x&&&&&&&&&&&&&&&&\\
    \hline
    พัฒนาโปรแกรมให้สามารถวิเคราะห์ &x&x&x&x&&&&&&&&&&&&&&&&\\
	อารมณ์ของลูกค้า &&&&&&&&&&&&&&&&&&&&\\
	\hline
    พัฒนาโปรแกรมให้สามารถวิเคราะห์ &&&&&x&x&x&x&x&x&x&x&x&x&x&&&&&\\
	อารมณ์ของลูกค้าที่มีต่อพนักงาน  &&&&&&&&&&&&&&&&&&&&\\
    \hline
    ทำรายงาน &&&&&x&x&x&x&x&x&x&x&x&x&x&x&x&x&x&x\\
    \hline
   
  \end{tabular}
  \caption{ตารางการดำเนินโครงงาน (ต่อ)}
  \label{fig:table2}
  \hrulefill
\end{table}
\end{landscape} 
