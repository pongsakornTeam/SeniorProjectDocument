%%%%%%%%%%%%%%%%%%%%%%%%%%%%%% 
%
% Thammasat University Thesis Template
% 1. Thesis.tex (this file)  edit:
%    a. Thesis Data, e.g. title, author
%    b. Committee members on the Signatures page
%    c. Chapters
%    d. Appendices
% 
% 2. The main content of the thesis should be in the following files:
%    a. ack.tex
%    b. abstract.tex
%    c. acronyms.tex 
%    d. chapter1.tex, chapter2.tex, ...
%    e. appendixA.tex, appendixB.tex, ...
%    f. refs.bib 
%
%%%%%%%%%%%%%%%%%%%%%%%%%%%%%%%%%%%%%%%%%%%%%%%%%%%%%%%%%%%%%%%%%%%%%%%%%%%%%% 
\documentclass[12pt,a4paper,oneside]{book}

%%%%%%%%%%%%%%%%%%%%%%%%%%%%%%%%%%%%%%%%%%%%%%%%%%%%%%%%%%%%%%%%%%%%%%%%%%%%%% 
%
% Packages
% (add/remove packages as needed)
%
%%%%%%%%%%%%%%%%%%%%%%%%%%%%%%%%%%%%%%%%%%%%%%%%%%%%%%%%%%%%%%%%%%%%%%%%%%%%%% 

% Required packages for thesis layout (do not remove)
\usepackage{TUthesis} 

% Other packages (add/delete packages as desired)
%\input{listing}
\input{lstlisting}
%\usepackage{graphicx}

\usepackage{svn}
\usepackage[plainpages=false]{hyperref}
\usepackage{url}
\usepackage{amsmath}
\usepackage{amssymb}
\usepackage{amsthm}
\usepackage{multirow}
\usepackage{caption}
\usepackage{subcaption}
\usepackage{lscape}
\usepackage{float}
\usepackage{tikz}
\usepackage{mwe}
\usetikzlibrary{arrows,shadows,positioning,shapes,calc,shapes.geometric}
\usepackage{graphicx}
\usepackage{multirow}

%\usepackage[superscript]{cite}
\newfontfamily{\tt}[Scale=0.8]{Courier New}

%\newfontfamily{\thaifont}[Scale=0.8]{Courier New}
\newfontfamily{\thsarabun}{THSarabunNew}
%\newfontfamily{\thaifont}[Path = /usr/local/share/fonts/]{THSarabunNew.ttf}
%\newfontfamily{\haiBF}[Path = /usr/local/share/fonts/]{THSarabunNew Bold.ttf}
%\newfontfamily{\thaiIT}[Path = /usr/local/share/fonts/]{THSarabunNew Italic.ttf}
%\usepackage{multibib}
%\newcites{book}{หนังสืออ้างอิง}
%\newcites{journal}{วารสารอ้างอิง}

%% figures that you want to include
%\graphicspath{{./}{./figures/}}
\graphicspath{{Figures/}}


%%%%%%%%%%%%%%%%%%%%%%%%%%%%%%%%%%%%%%%%%%%%%%%%%%%%%%%%%%%%%%%%%%%%%%%%%%%%%% 
%
% Thesis Data
% (edit each field)
%
%%%%%%%%%%%%%%%%%%%%%%%%%%%%%%%%%%%%%%%%%%%%%%%%%%%%%%%%%%%%%%%%%%%%%%%%%%%%%% 
% Thesis Title in mixed case
\newcommand{\thesistype}{Thesis} %Thesis for Master or Dissertation for PhD
\newcommand{\thaititle}{การวิเคราะห์วีดีทัศน์สำหรับธุรกิจค้าปลีก}
\newcommand{\engtitle}{Video Analytics for retail operations}

% Student name (author of the thesis)
\newcommand{\thaiauthorc}{นายรัฐสรรค์ เอกอัครชวกิตติ์}
\newcommand{\thaiauthorp}{นายพงศกร โกสมิง}
\newcommand{\engauthorc}{Mr. Rattasan Aekaukkharachawakit }
\newcommand{\engauthorp}{Mr. Pongsakorn Kosaming}


% Degree that this thesis is for
\newcommand{\thaidegree}{วิศวกรรมศาสตรบัณฑิต}
\newcommand{\engdegree}{Bachelor of Engineering}

\newcommand{\thaimajor}{วิศวกรรมคอมพิวเตอร์}
\newcommand{\engmajor}{Computer Engineering}


\newcommand{\thaidate}{29 กันยายน }

% Academic year for thesis submission
\newcommand{\thaiyear}{2560}
\newcommand{\engyear}{2017}

% Institute/department
\newcommand{\thaifaculty}{คณะวิศวกรรมศาสตร์}
\newcommand{\engfaculty}{Faculty of Engineering}


\newcommand{\thaikeyword}{ฟิชชิง การยืนยันเซิร์ฟเวอร์ การยืนยันผู้ใช้}
\newcommand{\engkeyword}{Phishing, server authentication, client authentication, certificates}

\newcommand{\chairman}{ผู้ช่วยศาสตราจารย์ ดร. นพพร ลีปรีชานนท์ }
\newcommand\advisor{รองศาสตราจารย์ ดร. ชนาทิพย์ นามเปรมปรีดิ์}
\newcommand\engadvisor{Assoc. Prof. Chanathip Namprempre, Ph.D.}

\newcommand\memberone{Metthew N. Dailey, Ph.D.}
\newcommand\membertwo{รองศาสตราจารย์ ดร.จาตุรงค์ ตันติบัณฑิต}
\newcommand\dean{รศ. ดร. }
\setlength{\parindent}{0.8in}
\def\shortbib{0}
\newcommand*{\captionsource}[2]{%
  \caption[{#1}]{%
    #1%
    \\\hspace{\linewidth}%
    \textbf{แหล่งที่มา:} #2%
  }%
}

%%%%%%%%%%%%%%%%%%%%%%%%%%%%%%%%%%%%%%%%%%%%%%%%%%%%%%%%%%%%%%%%%%%%%%%%%%%%%% 
%
% Title Page
% (do not edit)
%
%%%%%%%%%%%%%%%%%%%%%%%%%%%%%%%%%%%%%%%%%%%%%%%%%%%%%%%%%%%%%%%%%%%%%%%%%%%%%%
%\hyphen{authen-tication}

\input{macros}

\linespread{1.8}


\begin{document}
\frontmatter
\addtocontents{toc}{\contheading}
\setlength{\parskip}{0pt plus1pt minus1pt}

\pagestyle{empty}
\begin{center}
\vspace*{6mm}
	\begin{center}
		\includegraphics[width=1in]{tu-logo-bw.jpg}\\
	\end{center}
	\vspace{1cm}
%		\vspace*{\fill}
%
\begin{spacing}{1.4}
\textbf{\huge\thaititle}
\end{spacing}
\vspace*{35mm}
\textbf{โดย}\\
\vspace*{10mm}
\textbf{\thaiauthorc}\\
\textbf{\thaiauthorp}\\
\vfill
\begin{spacing}{1.3}
\textbf{
โครงงานนี้เป็นส่วนหนึ่งของการศึกษาตามหลักสูตร\\
\thaidegree \\ สาขา\thaimajor\\
\thaifaculty \ มหาวิทยาลัยธรรมศาสตร์  \\
ปีการศึกษา \thaiyear\\ ลิขสิทธิ์ของมหาวิทยาลัยธรรมศาสตร์
}
\end{spacing}
\end{center}

\newpage


%%%%%%%%%%%%%%%%%%%%%%%%%%%%%%%%%%%%%%%%%%%%%%%%%%%%%%%%%%%%%%%%%%%%%%%%%%%%%% 
%
% Signatures
%
%%%%%%%%%%%%%%%%%%%%%%%%%%%%%%%%%%%%%%%%%%%%%%%%%%%%%%%%%%%%%%%%%%%%%%%%%%%%%% 
\pagestyle{empty}
\begin{center}
\vspace*{40mm}
\begin{spacing}{1.4}
\textbf{\huge\thaititle}
\end{spacing}
\vspace*{35mm}
\textbf{โดย}\\
\vspace*{10mm}
\textbf{\thaiauthorc}\\
\textbf{\thaiauthorp}\\
\vfill
\begin{spacing}{1.3}
\textbf{
โครงงานนี้เป็นส่วนหนึ่งของการศึกษาตามหลักสูตร\\
\thaidegree \\ สาขา\thaimajor\\
\thaifaculty \ มหาวิทยาลัยธรรมศาสตร์  \\
ปีการศึกษา \thaiyear\\ ลิขสิทธิ์ของมหาวิทยาลัยธรรมศาสตร์
}
\end{spacing}
\end{center}
\newpage

\pagestyle{empty}
\begin{center}
\vspace*{40mm}
\begin{spacing}{1.4}
\textbf{\huge\engtitle}
\end{spacing}
\vspace*{35mm}
\textbf{BY}\\
\vspace*{10mm}
\textbf{\engauthorc}\\
\textbf{\engauthorp}\\
\vfill
\begin{spacing}{1.3}
\textbf{
A PROJECT SUBMITTED IN PARTIAL FULFILLMENT OF THE\\
REQUIREMENTS FOR THE DEGREE OF BACHELOR OF ENGINEERING \\
IN COMPUTER ENGINEERING\\
FACULTY OF ENGINEERING\\ THAMMASAT UNIVERSITY  \\
ACADEMIC YEAR \engyear\\ COPYRIGHT OF THAMMASAT UNIVERSITY
}
\end{spacing}
\end{center}
\newpage
%====

\vspace*{2mm}
\begin{center}
มหาวิทยาลัยธรรมศาสตร์\\
\textnormal{\thaifaculty} \\
\vspace*{10mm}
โครงงาน\\
\vspace*{10mm}
ของ\\
\vspace*{10mm}
\thaiauthorc\\
\thaiauthorp\\
\vspace*{10mm}
เรื่อง\\
\vspace*{10mm}
\thaititle \\
\vspace*{10mm}
\begin{spacing}{1.3}
ได้รับการตรวจสอบและอนุมัติ ให้เป็นส่วนหนึ่งของการศึกษาตามหลักสูตร\\
\thaidegree
\end{spacing}
\vspace*{8mm}
เมื่อวันที่ \thaidate พ.ศ. 2560 \\
\end{center}
\vspace*{2mm}
%\vfill
%~Approved as to style and content by\\[-2ex]
\begin{table}[h]
\setlength{\tabcolsep}{2.7ex}
\begin{tabular}{ccc}
							
อาจารย์ที่ปรึกษาโครงงาน				& \\ \cline{2-2}
							 		&(\advisor)\\[3ex]
หัวหน้าภาควิชาวิศวกรรมไฟฟ้าและคอมพิวเตอร์    	& \\ \cline{2-2}
									& (\chairman)\\[3ex]
							
\end{tabular}
\end{table}
\newpage

\setlength{\parindent}{0.8in}
\pagestyle{plain}
\setcounter{page}{1}
  \renewcommand{\thepage}{(\arabic{page})} %\setcounter{page}{-1}

%%%%%%%%%%%%%%%%%%%%%%%%%%%%%%%%%%%%%%%%%%%%%%%%%%%%%%%%%%%%%%%%%%%%%%%%%%%%%% 
%
% Abstract
% (do not edit here; instead edit the file abstract.tex)
%
%%%%%%%%%%%%%%%%%%%%%%%%%%%%%%%%%%%%%%%%%%%%%%%%%%%%%%%%%%%%%%%%%%%%%%%%%%%%%% 
\iffalse
  \begin{table}[h]
\setlength{\tabcolsep}{0.7ex}
\begin{tabular}{ccc}
%===========================================================================
หัวข้อโครงงาน			    		&& \thaititle\\
ชื่อผู้เขียน				  		&& \thaiauthorc\\ 
&& \thaiauthorp \\
ชื่อปริญญา						&& \thaidegree \\
สาขาวิชา/คณะ/มหาวิทยาลัย	&& \thaimajor\ \thaifaculty\ มหาวิทยาลัยธรรมศาสตร์ \\
อาจารย์ที่ปรึกษาโครงงาน    		&& \advisor\\
ปีการศึกษา						&& \thaiyear \\
\end{tabular}
\end{table}
\chapter*{บทคัดย่อ}
\input{abstractth}
\addcontentsline{toc}{section}{บทคัดย่อ}
\addtocontents{toc}{\protect\vspace{\baselineskip}}

\begin{center} {\Large บทคัดย่อ} \end{center}
\input{abstractth}

\vspace{8mm}
\noindent \textbf{คำสำคัญ:} \thaikeyword

\newpage
\begin{table}[h]
\setlength{\tabcolsep}{0.7ex}
\begin{tabular}{ccc}
%===========================================================================
\thesistype \ Title			    		&& \engtitle\\
Author				  		&& \engauthorc \\
&& \engauthorp\\
Degree						&& \engdegree \\
Major Field/Faculty/University		&& \engmajor \\
&& \engfaculty \\
&& Thammasat University \\
\thesistype \ Advisor			    		&& \engadvisor\\
Academic Year						&& \engyear \\
\end{tabular}
\end{table}
\addcontentsline{toc}{section}{Abstract}
\addtocontents{toc}{\protect\vspace{\baselineskip}}

\begin{center} \textbf{\Large ABSTRACT} \end{center}
\input{abstract}

 \vspace{8mm}
 \noindent \textbf{keyword:} \engkeyword

\fi
%%%%%%%%%%%%%%%%%%%%%%%%%%%%%%%%%%%%%%%%%%%%%%%%%%%%%%%%%%%%%%%%%%%%%%%%%%%%%% 
%
% Acknowledgement
% (do not edit here; instead edit the file ack.tex)
%
%%%%%%%%%%%%%%%%%%%%%%%%%%%%%%%%%%%%%%%%%%%%%%%%%%%%%%%%%%%%%%%%%%%%%%%%%%%%%% 
%\chapter*{กิตติกรรมประกาศ \\[-1em]}
%\input{ack}
%\addcontentsline{toc}{section}{กิตติกรรมประกาศ}
%\addtocontents{toc}{\protect\vspace{\baselineskip}}
\newpage


%%%%%%%%%%%%%%%%%%%%%%%%%%%%%%%%%%%%%%%%%%%%%%%%%%%%%%%%%%%%%%%%%%%%%%%%%%%%%% 
%
% Table of Contents, Figures and Tables
% (comment/uncomment any tables/lists that you desire)
%
%%%%%%%%%%%%%%%%%%%%%%%%%%%%%%%%%%%%%%%%%%%%%%%%%%%%%%%%%%%%%%%%%%%%%%%%%%%%%% 
%\addcontentsline{toc}{section}{สารบัญ}
\tableofcontents
\newpage
\addcontentsline{toc}{section}{สารบัญรูป}
\addtocontents{toc}{\protect\vspace{\baselineskip}}
\listoffigures
\newpage
\addcontentsline{toc}{section}{สารบัญตาราง}
\addtocontents{toc}{\protect\vspace{\baselineskip}}
\listoftables
%\addcontentsline{toc}{section}{List of Listings}
%\lstlistoflistings
%\newpage


%%%%%%%%%%%%%%%%%%%%%%%%%%%%%%%%%%%%%%%%%%%%%%%%%%%%%%%%%%%%%%%%%%%%%%%%%%%%%% 
%
% List of Acronyms
% (do not edit here; instead edit the file acronyms.tex)
%
%%%%%%%%%%%%%%%%%%%%%%%%%%%%%%%%%%%%%%%%%%%%%%%%%%%%%%%%%%%%%%%%%%%%%%%%%%%%%% 

\addcontentsline{toc}{section}{สัญลักษณ์และคำย่อ}
\chapter*{สัญลักษณ์และคำย่อ \\[-1em]}
\input{acronyms}
\newpage


%%%%%%%%%%%%%%%%%%%%%%%%%%%%%%%%%%%%%%%%%%%%%%%%%%%%%%%%%%%%%%%%%%%%%%%%%%%%%% 
%
% Chapters
% (edit the files chapter1.tex, chapter2.tex, ...)
%
%%%%%%%%%%%%%%%%%%%%%%%%%%%%%%%%%%%%%%%%%%%%%%%%%%%%%%%%%%%%%%%%%%%%%%%%%%%%%% 
\mainmatter
\clearpage
%\input{test}
\chapter{บทนำ}
\label{intro}
%%%%%%%%%%%%%%%%%%%%%%%%%%%%%%%%%%%%%%%%%%%%%%%%%%%%%%%%%%%%%%%%%%%%%%%%%%%%%% 
%
\section{ที่มาและความสำคัญ}
ในปัจจุบันมีการบริการมากมายเกิดขึ้นในชีวิตประจำวันของมนุษย์ ซึ่งการบริการในการขายสินค้าอย่างหนึ่งต้องมีการพูดคุยหรือติดต่อกับลูกค้า ซึ่งหากจะกล่าวถึงการบริการแบบร้านค้าที่มีพนักงานขายติดต่อสื่อสารกับลูกค้าที่มารับบริการโดยตรงแล้ว ความพึงพอใจการบริการที่ได้รับ ก็สามารถแสดงออกในรูปแบบของ สีหน้าอารมณ์ คำพูด และกริยาท่าทางหลังจากได้รับบริการ

ความพึงพอใจต่อการบริการ จึงเป็นสาระสำคัญในการขายสินค้าหรือบริการ ซึ่งสามารถวัดระดับคุณภาพการบริการได้จากลูกค้าที่กำลังได้รับการบริการอยู่ในขณะนั้นเอง  ซึ่งเป็นเหตุผลที่โครงงานนี้เสนอการสร้างซอฟต์แวร์ที่สามารถตรวจจับอารมณ์ของลูกค้า ซึ่งถือว่าเป็นสิ่งที่ระบุความพึงพอใจของลูกค้าได้ในระดับหนึ่ง โดยวิธีการทำงาน คือ นำข้อมูลภาพเคลื่อนไหวที่ได้รับจากกล้องวงจรปิดภายในร้านค้า มาใช้เป็นแหล่งข้อมูลเพื่อให้ซอฟต์แวร์วิเคราะห์ข้อมูลนั้น แล้วบันทึกผลลัพธ์ออกมาซึ่งเป็นความพึงพอใจของลูกค้าทุกคนที่ปรากฎในภาพเคลื่อนไหวจากกล้องวงจรปิด ณ เวลาขณะนั้นได้  และในการที่จะยกระดับหรือปรับปรุงการบริการ จำเป็นจะต้องมีการบันทึกข้อมูลการบริการของพนักงาน เพื่อให้ทราบว่าลูกค้ากำลังได้รับการบริการจากพนักงานคนใดและมีระดับความพึงพอใจต่อการบริการในระดับใดด้วย

เมื่อมีการใช้งานซอฟต์แวร์ที่ทำการวิเคราะห์ความพึงพอใจของลูกค้าแล้ว จึงต้องมีการใช้ภาพเคลื่อนไหวจากกล้องวงจรปิดเพื่อเป็นแหล่งข้อมูลขาเข้าให้ซอฟต์แวร์วิเคราะห์ ซึ่งภาพเคลื่อนไหวที่ได้จะต้องมีรูปใบหน้าของลูกค้าอย่างชัดเจน โครงงานนี้จะพัฒนาต่อยอดซอฟต์แวร์ที่สามารถวิเคราะห์ได้ทีละหนึ่งภาพเคลื่อนไหวที่เป็นข้อมูลขาเข้าจากกล้องวงจรปิดเพียงหนึ่งมุมเท่านั้น ซึ่งโครงงานนี้จะอ้างอิงจากวิทยานิพนธ์และซอฟต์แวร์ของคุณ Nabil Tahmidul Karimv \cite{nabil} และSanjana Jain \cite{sanjana} โดยจะทำการพัฒนาและประยุกต์ให้ใช้สามมุมกล้องภายในร้านค้าในการวิเคราะห์ภาพเคลื่อนไหวทั้งสามมุม เพื่อตรวจจับอารมณ์ที่แสดงออกความพึงพอใจของลูกค้าที่มีต่อพนักงานขายที่กำลังบริการอยู่ ณ ขณะนั้น โดยใช้เวลาเป็นตัวเชื่อมโยงความสัมพันธ์ของภาพเคลื่อนไหวทั้งสามมุมจากสามกล้องวงจรปิดในเวลาเดียวกัน

เนื่องจากการทำงานของซอฟต์แวร์ที่ถูกพัฒนาขึ้นมานั้นสามารถจดจำลูกค้าและตรวจจับความพึงพอใจจากอารมณ์ โครงงานนี้จึงมีการนำมาพัฒนาระบบให้ตรวจจับความพึงพอใจของลูกค้าต่อพนักงานบริการ จากภาพเคลื่อนไหวหลายจากกล้องวงจรปิดหลายมุมขึ้นมา เพื่อทำให้สามารถใช้เป็นข้อมูลในปรับปรุงการบริการได้อย่างมีประสิทธิภาพและเป็นการนำความก้าวหน้าทางเทคโนโลยีมาใช้ประโยชน์ในทางการค้าอย่างมีประสิทธิภาพ โดยนำทฤษฎีการประมวลผลภาพดิจิตอล (digital image processing)  และการเรียนรู้ของเครื่องจักร (machine learning)  ในประเภทการเรียนรู้แบบกำกับดูแล (supervised leaning)  ซึ่งเป็นสาขาหนึ่งของปัญญาประดิษฐ์ (artificial intelligence)  ที่พัฒนามาจากการศึกษารู้จำรูปแบบ เพื่อสร้างขั้นตอนวิธีที่สามารถประมวลผลภาพเคลื่อนไหวที่เกิดจากการนำภาพดิจิตอลหลายๆภาพมาเรียงต่อกันให้สามารถแสดงผลลัพธ์ที่ต้องการได้ นั่นคือ  ระดับความพึงพอใจของลูกค้าต่อพนักงานบริการที่กำลังบริการอยู่
 
\section{วัตถุประสงค์}
\begin{enumerate}[label=1.2.\arabic*]
%\setlength{\itemindent}{20pt}
\item{เพื่อศึกษาทฤษฎีการวิเคราะห์และประมวลผลภาพเคลื่อนไหวที่มีภาพบุคคลคนเดียวกันที่ปรากฏพร้อมกันในกล้องวงจรปิด เพื่อออกแบบระบบให้มีความสามารถในการตรวจจับและระบุว่าเป็นบุคคลเดียวกัน}
\item{เพื่อพัฒนาขั้นตอนวิธีเพื่อให้ตรวจจับความพึงพอใจของลูกค้าที่มีต่อพนักงานที่กำลังบริการในขณะนั้น}
\end{enumerate}

\section{ขอบเขตการดำเนินงาน}
\begin{enumerate}[label=1.3.\arabic*]
%\setlength{\itemindent}{20pt}
\item{กลุ่มตัวอย่างที่ใช้ในการวิจัย ได้แก่ ภาพเคลื่อนไหวที่ได้จากกล้องวงจรปิดที่ร้าน
  Hom Krun สาขา AIT}
\item{ออกแบบและพัฒนาขั้นตอนวิธีให้สามารถตรวจจับความพึงพอใจของลูกค้าที่มีต่อพนักงานที่กำลังให้บริการในขณะนั้นได้}
\end{enumerate}

\section{ขั้นตอนการดำเนินงาน}
\begin{enumerate}[label=1.4.\arabic*]
%\setlength{\itemindent}{20pt}
\item{ศึกษาทฤษฎีเกี่ยวกับการรู้จำของเครื่อง (machine learning)}
\item{เก็บข้อมูลภาพเคลื่อนไหวจากกล้องวงจรปิดที่ร้าน
  Hom Krun สาขา AIT}
\item{วิเคราะห์ภาพเคลื่อนไหวและนำภาพใบหน้าพนักงานมาทำเป็นภาพนิ่งของแต่ละบุคคลเพื่อนำมาหาลักษณะเด่น (features)}
\item{ทำการนำภาพนิ่งที่มีรูปใบหน้าของพนักงานเพื่อใช้เป็นข้อมูลเข้าและลักษณะเด่น เพื่อมาทำการฝึกฝนพัฒนาระบบที่สามารถจำแนกประเภทด้วยการเรียนรู้แบบกำกับดูแล (supervised learning classifier)}
\item{สกัดลักษณะเด่นให้มีประสิทธิภาพสูงสุดโดยวิเคราะห์ทั้งความถูกต้องและแม่นยำ (accuracy)}
\item{สร้างระบบที่มีการระบุตัวตนของบุคคลคนเดียวกันจากหลายภาพเคลื่อนไหว}
\item{เก็บข้อมูลความพึงพอใจของลูกค้าที่แสดงออกต่อพนักงานคนหนึ่งๆ}
\end{enumerate}

\section{ผลที่คาดว่าจะได้รับ}
\begin{enumerate}[label=1.5.\arabic*]
%\setlength{\itemindent}{20pt}
\item{ได้ขั้นตอนวิธีและซอฟท์แวร์ในการจดจำพนักงานและวิเคราะห์ความพึงพอใจของลูกค้าระหว่างบริการได้}
\item{ช่วยในการปรับปรุงหรือยกระดับการบริการให้มีประสิทธิภาพและรวดเร็วมากยิ่งขึ้น}
\end{enumerate}
\newpage
\begin{landscape}
\section{ตารางการดำเนินงาน}
\begin{table}[h!]
 \centering
  \begin{tabular}{|l|c|c|c|c|c|c|c|c|c|c|c|c|c|c|c|c|c|c|c|c|}
  
    \hline
    \multirow{2}{*}{\textbf{แผนการดำเนินงาน}}
    & \multicolumn{4}{c|}{สิงหาคม} 
    & \multicolumn{4}{c|}{กันยายน} 
    & \multicolumn{4}{c|}{ตุลาคม}
    & \multicolumn{4}{c|}{พฤศจิกายน}
    & \multicolumn{4}{c|}{ธันวาคม}\\
    \cline{2-21}
    & 1 & 2 & 3 & 4
    & 1 & 2 & 3 & 4
    & 1 & 2 & 3 & 4
    & 1 & 2 & 3 & 4
    & 1 & 2 & 3 & 4\\
    \hline
    
    ศึกษาทฤษฎีเกี่ยวกับ &x&x&x&x&&&&&&&&&&&&&&&& \\
    การรู้จำของเครื่อง  &&&&&&&&&&&&&&&&&&&&\\
    \hline
    พัฒนาโปรแกรมให้สามารถ  &&&&&x&&&&&&&&&&&&&&&\\
    ตรวจจับใบหน้าของลูกค้า &&&&&&&&&&&&&&&&&&&&\\
     \hline
     ศึกษาเกี่ยวกับ  &&&&&&x&x&x&x&&&&&&&&&&&\\
     camera calibration &&&&&&&&&&&&&&&&&&&&\\
    \hline
    ศึกษาเกี่ยวกับ &&&&&x&x&x&x&x&x&x&x&x&&&&&&&\\
    caffe model &&&&&&&&&&&&&&&&&&&&\\
    \hline
    พัฒนาโปรแกรมให้สามารถจดจำ  &&&&&&&&&&&&&&x&x&x&x&&&\\
    ลักษณะของพนักงาน  &&&&&&&&&&&&&&&&&&&&\\
    \hline
    ทำรายงาน &&&&&x&x&x&x&x&x&x&x&x&x&x&x&x&x&x&x\\
    \hline
    
  \end{tabular}
  \caption{ตารางการดำเนินโครงงาน}
  \label{fig:table}
\hrulefill
\end{table}
%\end{landscape}
%\newpage
%%%%%%%%%%%%%%%%%%%%%%%%%%%%%%%%%%%%%%%%%%%%%%%%%%%%%%%%
%\begin{landscape}
\begin{table}[h!]
 \centering
  \begin{tabular}{|l|c|c|c|c|c|c|c|c|c|c|c|c|c|c|c|c|c|c|c|c|c|}
  
    \hline
    \multirow{2}{*}{\textbf{แผนการดำเนินงาน}}
    & \multicolumn{4}{c|}{มกราคม} 
    & \multicolumn{4}{c|}{กุมภาพันธ์} 
    & \multicolumn{4}{c|}{มีนาคม}
    & \multicolumn{4}{c|}{เมษายน}
    & \multicolumn{4}{c|}{พฤษภาคม}\\
    \cline{2-21}
    & 1 & 2 & 3 & 4
    & 1 & 2 & 3 & 4
    & 1 & 2 & 3 & 4
    & 1 & 2 & 3 & 4
    & 1 & 2 & 3 & 4\\
	 \hline
    ศึกษาวิธีการวิเคราะห์อารมณ์ของลูกค้า  &x&x&x&x&&&&&&&&&&&&&&&&\\
    \hline
    พัฒนาโปรแกรมให้สามารถวิเคราะห์ &x&x&x&x&&&&&&&&&&&&&&&&\\
	อารมณ์ของลูกค้า &&&&&&&&&&&&&&&&&&&&\\
	\hline
    พัฒนาโปรแกรมให้สามารถวิเคราะห์ &&&&&x&x&x&x&x&x&x&x&x&x&x&&&&&\\
	อารมณ์ของลูกค้าที่มีต่อพนักงาน  &&&&&&&&&&&&&&&&&&&&\\
    \hline
    ทำรายงาน &&&&&x&x&x&x&x&x&x&x&x&x&x&x&x&x&x&x\\
    \hline
   
  \end{tabular}
  \caption{ตารางการดำเนินโครงงาน (ต่อ)}
  \label{fig:table2}
  \hrulefill
\end{table}
\end{landscape} 

\newpage
\chapter{วรรณกรรมและงานวิจัยที่เกี่ยวข้อง}
\label{literature}
%%%%%%%%%%%%%%%%%%%%%%%%%%%%%%%%%%%%%%%%%%%%%%%%%%%%%%%%%%%%%%%%%%%%%%%%%%%%%% 
%
\section{Neural network \cite{Neuron}}
Neural Network หรือโครงข่ายประสาทเทียมเป็นแบบจำลองทางคณิตศาสตร์
ที่ใช้ในการจำลองการทำงานของโครงข่ายประสาทของมนุษย์โครงข่ายประสาทเทียม ประกอบด้วย เซลล์ประสาทเทียมเรียกว่าโหนด (node) และจุดประสานประสาทที่ทำหน้าที่เชื่อมระหว่างเซลล์ประสาทเรียกกว่าค่าน้ำหนัก (weights) ซึ่งโครงข่ายประสาทเทียมจะแบ่งเป็น 3 ชั้นซึ่งประกอบด้วย ชั้นนำเข้า (input layer) ชั้นซ่อน (hidden layer) ชั้นส่งออก (output layer) ดังรูปที่ \ref{fig:neuron-network}

ชั้นนำเข้า (input layer) เป็นชั้นที่รอรับข้อมูลเพื่อป้อนเข้าสู่โครงข่ายประสาทเทียม

ชั้นซ่อน (hidden layer) เป็นชั้นที่เพิ่มประสิทธิภาพในการจัดกลุ่มข้อมูลโดยสามารถคำนวณได้ดังสมการ
\[y_j = f(\sum_{i=1}^N x_i w_{ij} + \theta_j)\]
โดยที่ \(y_j\) คือ ผลลัพธ์ในชั้นซ่อน หรือข้อมูลส่งออกในชั้นซ่อนที่โหนด j

\(x_i\) คือ ข้อมูลนำเข้าที่โหนด i ในชั้นนำเข้า

\(w_{ij}\) คือ น้ำหนักบนเส้นเชื่อมระหว่างโหนดที่ i และ โหนดที่ j ในชั้นซ่อน

\(\theta_j\) คือค่าโน้มเอียงของโหนด j ในชั้นซ่อน

โดยภายในโหนดจะมีส่วนประกอบคือ
 ข้อมูลป้อนเข้า (input) ,
 ข้อมูลส่งออก (output) ,
 ค่าน้ำหนัก (weight) และ
 ฟังก์ชั่นในการคำนวน ดังรูปที่ \ref{fig:neuron}โดยตัวอย่างฟังก์ชั่นในการคำนวณเช่น ฟังก์ชันซิกมอยด์
 
   Sigmoid เป็นฟังก์ชั่นที่สร้างตัวเลขในช่วงระหว่าง 0 ถึง 1 โดยตัวเลขที่มีค่าไปทางบวกมาก จะเข้าใกล้ 1 และตัวเลขที่มีค่าไปทางลบมากจะเข้าใกล้ 0 ตามสมการ
    \[\sigma(x) = \frac{1}{1+e^x}\]
    โดยที่ \(\sigma(x)\) คือ sigmoid function
    
    \(x\) คือผลลัพธ์ที่ได้จาก \(\sum_{i=1}^N x_i w_{ij} + \theta_j\)

    \(e\) มีค่าประมาณ 2.71828
 
        \subsection{Back Propagation}
        โดยการฝึกโคร่งข่ายประสาทเทียม (Neuron network) ให้สามารถรู้จำได้จะใช้อัลกอริทึม
        Back Propagation โดยจะมีวิธีการดังนี้ 
        \begin{enumerate}[label=2.1.\arabic*]
      \item{สุ่มค่าน้ำหนัก(weight)}ให้กับแต่ละเซลล์ (neuron)
      \item{กำหนดค่าโน้มเอียง (bias) }เป็นค่าเริ่มต้าในขั้นตอนแรกโดยจะมีค่าเท่ากับ 1 หรือสุ่มขึ้นมา
      \item{นำข้อมูลเข้าไปในโคร่งข่ายประสาทเทียมผ่านชั้นนำเข้า (input layer) }
      \item{คำนวนค่านำเข้า(input)และค่าส่งออก(output)ในแต่ละชั้น (layer) }
      \item{คำนวนค่าคลาดเคลื่อนในแต่ละชั้น (layer) }โดยคำนวนจากชั้นส่งออก (output layer) จากนั้นจึงคำนวนชั้นถัดมาโดยค่าคลาดเคลื่อนจะคิดได้จากผลต่างของค่าส่งออก (output) เทียบกับค่าเป้าหมาย (target)
      \item{ปรับค่าน้ำหนักและค่าโน้มเอียงใหม่จากค่าคลาดเคลื่อนที่ได้}
      \item{ทำกระบวนการเดิมไปเรื่อยๆเพื่อเพิ่มความแม่นยำในการรู้จำของเครื่อง} โดยจะหยุดทำเมื่อค่าคลาดเคลื่อนมีเพิ่มขึ้นหรือคงที่
        \end{enumerate} 
\begin{center}
  \begin{figure}[t]
   \captionsetup{justification=centering}
    \centering
    \includegraphics[width=3in]{neuron_model.jpeg}
  \captionsource{ภาพตัวอย่างการทำงานของเซลล์ประสาทเทียม (Neuron)}{http://cs231n.github.io/neural-networks-1/}
  \label{fig:neuron}
  \hrulefill
\end{figure}
 \end{center}
  \begin{center}
  \begin{figure}[t]
    \captionsetup{justification=centering}
    \centering
    \includegraphics[width=3in]{neural_net.jpeg}
  \captionsource{ภาพตัวอย่างการทำงานของโครงข่ายประสาทเทียม (Neural network)}{http://cs231n.github.io/neural-networks-1/}
  \label{fig:neuron-network}
  \hrulefill
\end{figure}
  \end{center}
 \section{Deep neural network} Deep neural network เป็นหัวข้อหนึ่งของ การเรียนรู้ของเครื่อง (machine learning) โดยจะมีลักษณะเหมือนกับโครงข่ายประสาทเทียม (neural network) เพียงแต่มีจำนวน hidden layer ที่มากกว่า ดังรูปที่ \ref{fig:deep-network} เนื่องในหลายครั้งการแก้ปัญหาที่ซับซ้อนมากจะไม่สามารถแก้ได้ด้วยการมี hidden layer น้อยๆได้
  \begin{center}
  \begin{figure}[t]
    \captionsetup{justification=centering}
    \centering
    \includegraphics[width=4in]{deep_network.png}
  \captionsource{ภาพตัวอย่างการทำงานของโครงข่ายประสาทเทียมเชิงลึก (Deep neural network)}{https://i.stack.imgur.com/1bCQl.png}
  \label{fig:deep-network}
 \hrulefill
\end{figure}
 \end{center}
\section{Convolutional neural network \cite{Deep}}
Convolutional neural network เป็นเทคนิคหนึ่งของ deep neural network โดยโครงสร้างของ convolution neural network นั้นจะประกอบด้วย 3 ชั้นคือ Convolution layers , Pooling layers , Fully connected layer ดังรูป \ref{fig:convo-network}

Convolution layers เป็นชั้นที่ทำหน้าที่เหมือนกับการสกัดลักษณะเด่น (features) ของข้อมูลป้อนเข้าโดยจะทำการเลือกขนาดของตัวกรอง (filter) เช่น 5x5x3 โดยเลข 3 จะแทน RGB

Pooling layers เป็นชั้นที่จะทำหน้าที่ลดขนาดของข้อมูลที่จะประมวลผลลง
โดยที่นิยมใช้คือ max pooling

Fully connected layer เป็นชั้นที่จะรวมค่าคำตอบที่ได้รับจาก pooling layers และ convolution layers เพื่อนำไปจำแนกประเภท

 \begin{center}
  \begin{figure}[t]
    \captionsetup{justification=centering}
    \centering
    \includegraphics[width=5in]{convo.png}
  \captionsource{ตัวอย่างการทำงานของ convolutional neural network}{https://www.clarifai.com/technology}
  \label{fig:convo-network}
  \hrulefill
\end{figure}
 \end{center}
\section{Caffe \cite{Caffe}}
Caffe เป็นเฟรมเวิรค์สำหรับ Convolution neural network ที่ได้รับความนิยมเนื่องจากทำให้สามารถ
พัฒนาได้รวดเร็วและมีความยืดหยุ่นในการใช้งานโดยจะแบ่งการทำงานเป็นชั้น (layer)ดังนี้
\subsection {vision layers} เป็นชั้นที่ข้อมูลนำเข้าและข้อมูลผลลัพธ์เป็นรูปภาพโดยจะดึงลักษณะเด่นของภาพเช่นสี,พื้นที่ของภาพโดยประกอบด้วย

Convolution layer จะรับข้อมูลนำเข้าเป็นรูปภาพและดึงลักษณะเด่นของรูปภาพออกมาด้วยตัวกรอง (filter) ออกมาเป็นผลลัพธ์

Pooling layer รวมผลลัพธ์ที่ได้ convolution layer

Local Response Normalize (LRN) วางกรอบค่าของตัวเลขที่ได้ให้อยู่ใน
ขอบเขตที่ต้องการเช่นให้อยู่ระหว่าง -1 ถึง 1 เพื่อให้เครื่องรับภาระในการประมวลผลลดลง
\subsection {loss layers} เป็นชั้นที่ทำการเรียนรู้และเปรียบเทียบผลลัพธ์ที่ได้จากโมเดลกับผลลัพธ์ที่กำหนดเพื่อปรับค่าน้ำหนักด้วยวิธีการ backpropagation ตัวอย่างของ loss layer

Softmax เป็นการคำนวณ logarithm ของความน่าจะเป็นตามสมการ
\[f_j(z)=\frac{e^{z_j}}{\Sigma_{k^{e^{z_k}}}}\]
โดยที่ \(f_j(z)\) คือ softmax function

\(e\) มีค่าประมาณ 2.71828

\(z_j  ,  z_k\) คือค่าคะแนนความคลาดเคลื่อนที่ได้จากชั้นก่อนหน้า

\subsection {activation layers} เป็นชั้นที่รับข้อมูลนำเข้าเพื่อประมวลผลโดยจะทำงานแบบเดียวกับนิวรอน
ตัวอย่างของ activation layers

Sigmoid เป็นฟังก์ชั่น เป็นฟังก์ชั่นที่สร้างตัวเลขในช่วงระหว่าง 0 ถึง 1 โดยตัวเลขที่มีค่าไปทางบวกมาก จะเข้าใกล้ 1 ตามสมการ
\[\sigma(x) = \frac{1}{1+e^x}\]
    โดยที่ \(\sigma(x)\) คือ sigmoid function
    
    \(x\) คือผลลัพธ์ที่ได้จาก \(\sum_{i=1}^N x_i w_{ij} + \theta_j\)

    \(e\) มีค่าประมาณ 2.71828

\subsection{data layers} เป็นชั้นที่ทำการรับข้อมูลเพื่อส่งไปเป็นข้อมูลนำเข้าให้กับ layer อื่น
\subsection {common layers}ตัวอย่างของ common layers มีดังนี้

linear product เป็นกระบวนการเชื่อมต่อโดยข้อมูลนำเข้าเป็นเวกเตอร์โดยผลลัพธ์จะเป็นเวกเตอร์ที่มีความลึกเท่ากับจำนวนลักษณะเด่น (features)ที่สกัดได้

dropout เป็นการป้องกันการเกิด overfitting โดยการปรับโครงสร้างของโหนดใน
โครงข่ายประสาทเทียม


\newpage

%%%%%%%%%%%%%%%%%%%%%%%%%%%%%%%%%%%%%%%%%%%%%%%%%%%%%%%%%%%%%%%%%%%%%%%%%%%%%% 
%
% References
% (\bibliography refers to your BibTeX .bib file, e.g. refs.bib)
%
%%%%%%%%%%%%%%%%%%%%%%%%%%%%%%%%%%%%%%%%%%%%%%%%%%%%%%%%%%%%%%%%%%%%%%%%%%%%%% 
\renewcommand\bibname{รายการอ้างอิง}

\backmatter
\cleardoublepage % needed to ensure TOC page number is correct
%% There is still a bug with the PDF hyperlink: clicking on the References
%% link in the PDF takes you to last page of previous chapter, not to start
%% of references
\addcontentsline{toc}{section}{รายการอ้างอิง}

%% These two commands are for when using BibTeX (i.e. most common)
%% Change 'refs' to be the name of your .bib file

%\bibliographystyle{plain} %vancouver
\bibliographystyle{unsrt}
\bibliography{crypto}

%%%%%%%%%%%%%%%%%%%%%%%%%%%%%%%%%%%%%%%%%%%%%%%%%%%%%%%%%%%%%%%%%%%%%%%%%%%%%% 
%
% Appendices
% (edit the files appendixA.tex, appendixB.tex, ...)
%
%%%%%%%%%%%%%%%%%%%%%%%%%%%%%%%%%%%%%%%%%%%%%%%%%%%%%%%%%%%%%%%%%%%%%%%%%%%%%% 

%% \appendix
%%  \part*{ภาคผนวก} 
%% \setlength{\parindent}{0.8in}
%% \input{appendixA}

\end{document}

